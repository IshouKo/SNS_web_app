\documentclass[a4paper,11pt]{jsarticle}


% 数式
\usepackage{amsmath,amsfonts}
\usepackage{bm}
% 画像
\usepackage[dvipdfmx]{graphicx}


\begin{document}

\title{Eraコードについて}
\author{平井希空}
\date{\today}
\maketitle

\section{app.pyについて}

\section{用語説明}
インスタンス (Instance) とは?\\
インスタンスは、「クラスという設計図から実際に作られた、具体的なもの」という意味です。\\
例えば、インスタンス (具体的なもの): 実際にユーザーを登録するときに、「user1 というユーザー名で、
test@example.com というメールアドレスのユーザー」という具体的なデータを持った実体が作られます。これが User クラスのインスタンスです。\\

コンテキスト (Context) とは?\\
コンテキストは、「ある操作が行われる際の、その状況や環境、背景情報」という意味です。
特にプログラミングでは、「コードが実行されるときに利用できる、特定の情報や状態」を指します。\\
たとえるなら、あなたが友達と話しているとき、「今、何の話をしてるんだっけ?」と内容(話題)を確認することがありますよね。
その「話題」や「場の雰囲気」がコンテキストです。\\


\end{document}